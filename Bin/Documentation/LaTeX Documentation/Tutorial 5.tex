\newcommand{\raul}{Ra\'ul}
\section{Tutorial 5: \raul{} Typesetting An Exam Paper}
One task of the department is to elucidate the past exam papers of those math courses. In this tutorial, we will see how \raul{} takes advantage of the package xjtluexam to typeset an exam paper.

\subsection{The exampaper Environment}
To begin an exam paper, \raul{} uses the exampaper environment. The environment itself gives no output, but provides several commands\footnote{In fact, the commands are not \emph{provided} by this environment. It merely does some initialization works so that these commands can function normally.}.

Then, \raul{} types \verb=\question= to start a question.
\begin{miniexammar}{.5\textandmarginlen}{
\begin{exampaper}
\question Justify whether $\infseries{n}{1}$ converges or not when $a_n \to 0$.
\question Evaluate
\[
\int x^5 e^{x^2} \dx
\]
\end{exampaper}
}
\begin{lstlisting}
\begin{exampaper}
\question Justify whether $\infseries{n}{1}$ converges or not if $a_n \to 0$.
\question Evaluate
\[
\int x^5 e^{x^2} \dx
\]
\end{exampaper}
\end{lstlisting}
\end{miniexammar}

The question counter can be referenced normally.
\begin{miniexammar}{.5\textandmarginlen}{
\begin{exampaper}
\question Justify whether $\infseries{n}{1}$ converges or not when $a_n \to 0$.
\question \label{ques:examques} Evaluate
\[
\int x^5 e^{x^2} \dx
\]
\end{exampaper}
A solution to Question \ref{ques:examques} is ...
}
\begin{lstlisting}
\begin{exampaper}
\question Justify whether $\infseries{n}{1}$ converges or not when $a_n \to 0$.
\question \label{ques:examques} Evaluate
\[
\int x^5 e^{x^2} \dx
\]
\end{exampaper}
A solution to Question \ref{ques:examques} is ...
\end{lstlisting}
\end{miniexammar}

Subquestions may be given, and the numbering for them is alphabetical.
\begin{miniexammar}{.5\textandmarginlen}{
\begin{exampaper}
\question Given the function $f=...$

\hspace{1.5em}(a) Evaluate $f(3)$

\hspace{1.5em}(b) Prove a proposition about $f$.
\end{exampaper}
}
\begin{lstlisting}
\begin{exampaper}
\question Given the function $f=...$
\subquestion Evaluate $f(3)$
\subquestion Prove a proposition about $f$.
\end{exampaper}
\end{lstlisting}
\end{miniexammar}

\subsection{Multiple Choice Questions}
Multiple choice questions (MCQs) appear in exam papers quite often. xjtluexam packages designs several environments and commands just for them.

To introduce the choices of a MCQ, \raul{} begins a choice environment. There are two choice environments: \emph{shortchoices} and \emph{longchoices}. As their name suggest, they are used to introduce short choices (multiple choices are in one line) and long choices (one choice occupies one line), respectively.

Inside one of the environments, a choice number is given by the \verb=\choice= command. The space after a choice is given by the command \verb=\choicespace=. \raul{} doesn't want a space at the end of the last choice, so he only uses the command between choices.
\begin{miniexammar}{.55\textandmarginlen}{
\begin{exampaper}
\question Evaluate
\[
\int_{-1}^1 x^2\cosh{x} \dx
\]
\begin{shortchoices}
\choicenumber $\frac{2\sinh{1}}{3}$ \choicespace
\choicenumber $-\frac{2\sinh{1}}{3}$ \choicespace
\choicenumber $\frac{2\cosh{1}}{3}$ \choicespace
\choicenumber $-\frac{2\cosh{1}}{3}$ 
\end{shortchoices}
\end{exampaper}
}
\begin{lstlisting}
\begin{exampaper}
\question Evaluate
\[\int_{-1}^1 x^2\cosh{x} \dx\]
\begin{shortchoices}
\choicenumber $\frac{2\sinh{1}}{3}$ \choicespace \choicenumber $-\frac{2\sinh{1}}{3}$ \choicespace \choicenumber $\frac{2\cosh{1}}{3}$ \choicespace \choicenumber $-\frac{2\cosh{1}}{3}$ 
\end{shortchoices}
\end{exampaper}
\end{lstlisting}
\end{miniexammar}

If \raul{} feels that the space is too tight, he can freely break a line.
\begin{miniexammar}{.42\textandmarginlen}{
\begin{exampaper}
\question Evaluate
\[
\int_{-1}^1 x^2\cosh{x} \dx
\]
\begin{shortchoices}
\choicenumber $\frac{2\sinh{1}}{3}$\choicespace
\choicenumber $-\frac{2\sinh{1}}{3}$ \\
\choicenumber $\frac{2\cosh{1}}{3}$\choicespace
\choicenumber $-\frac{2\cosh{1}}{3}$
\end{shortchoices}
\end{exampaper}
}
\begin{lstlisting}
\begin{exampaper}
\question Evaluate
\[\int_{-1}^1 x^2\cosh{x} \dx\]
\begin{shortchoices}
\choicenumber $\frac{2\sinh{1}}{3}$ \choicespace \choicenumber $-\frac{2\sinh{1}}{3}$ \\ \choicenumber $\frac{2\cosh{1}}{3}$ \choicespace \choicenumber $-\frac{2\cosh{1}}{3}$ 
\end{shortchoices}
\end{exampaper}
\end{lstlisting}
\end{miniexammar}

