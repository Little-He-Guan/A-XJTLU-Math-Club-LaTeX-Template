\newcommand{\raul}{Ra\'ul}
\section{Tutorial 5: \raul{} Typesetting An Exam Paper}
One task of the department is to elucidate the past exam papers of those math courses. In this tutorial, we will see how \raul{} takes advantage of the package xjtluexam to typeset an exam paper.

\subsection{The exampaper Environment}
To begin an exam paper, \raul{} uses the exampaper environment. The environment itself gives no output, but provides several commands\footnote{In fact, the commands are not \emph{provided} by this environment. It merely does some initialization works so that these commands can function normally.}.

Then, \raul{} types \verb=\question= to start a question.
\begin{miniexammar}{.5\textandmarginlen}{
\begin{exampaper}
\question Justify whether $\infseries{n}{1}$ converges or not when $a_n \to 0$.
\question Evaluate
\[
\int x^5 e^{x^2} \dx
\]
\end{exampaper}
}
\begin{lstlisting}
\begin{exampaper}
\question Justify whether $\infseries{n}{1}$ converges or not if $a_n \to 0$.
\question Evaluate
\[
\int x^5 e^{x^2} \dx
\]
\end{exampaper}
\end{lstlisting}
\end{miniexammar}

The question counter can be referenced normally.
\begin{miniexammar}{.5\textandmarginlen}{
\begin{exampaper}
\question Justify whether $\infseries{n}{1}$ converges or not when $a_n \to 0$.
\question \label{ques:examques} Evaluate
\[
\int x^5 e^{x^2} \dx
\]
\end{exampaper}
A solution to Question \ref{ques:examques} is ...
}
\begin{lstlisting}
\begin{exampaper}
\question Justify whether $\infseries{n}{1}$ converges or not when $a_n \to 0$.
\question \label{ques:examques} Evaluate
\[
\int x^5 e^{x^2} \dx
\]
\end{exampaper}
A solution to Question \ref{ques:examques} is ...
\end{lstlisting}
\end{miniexammar}

Subquestions may be given, and the numbering for them is alphabetical.
\begin{miniexammar}{.5\textandmarginlen}{
\begin{exampaper}
\question Given the function $f=...$

\hspace{1.5em}(a) Evaluate $f(3)$

\hspace{1.5em}(b) Prove a proposition about $f$.
\end{exampaper}
}
\begin{lstlisting}
\begin{exampaper}
\question Given the function $f=...$
\subquestion Evaluate $f(3)$
\subquestion Prove a proposition about $f$.
\end{exampaper}
\end{lstlisting}
\end{miniexammar}

\subsection{Multiple Choice Questions}
Multiple choice questions (MCQs) appear in exam papers quite often. xjtluexam package designs several environments and commands just for them.

To introduce the choices of a MCQ, \raul{} begins a choice environment. There are two choice environments: \emph{shortchoices} and \emph{longchoices}. As their name suggest, they are used to introduce short choices (multiple choices are in one line) and long choices (one choice occupies one line), respectively.

Inside one of the environments, a choice number is given by the \verb=\choice= command. The space after a choice is given by the command \verb=\choicespace=. \raul{} doesn't want a space at the end of the last choice, so he only uses the command between choices.
\begin{miniexammar}{.55\textandmarginlen}{
\begin{exampaper}
\question Evaluate
\[
\int_{-1}^1 x^2\cosh{x} \dx
\]
\begin{shortchoices}
\choicenumber $\frac{2\sinh{1}}{3}$ \choicespace
\choicenumber $-\frac{2\sinh{1}}{3}$ \choicespace
\choicenumber $\frac{2\cosh{1}}{3}$ \choicespace
\choicenumber $-\frac{2\cosh{1}}{3}$ 
\end{shortchoices}
\end{exampaper}
}
\begin{lstlisting}
\begin{exampaper}
\question Evaluate
\[\int_{-1}^1 x^2\cosh{x} \dx\]
\begin{shortchoices}
\choicenumber $\frac{2\sinh{1}}{3}$ \choicespace \choicenumber $-\frac{2\sinh{1}}{3}$ \choicespace \choicenumber $\frac{2\cosh{1}}{3}$ \choicespace \choicenumber $-\frac{2\cosh{1}}{3}$ 
\end{shortchoices}
\end{exampaper}
\end{lstlisting}
\end{miniexammar}

If \raul{} feels that the space is too tight, he can freely break the lines
\begin{miniexammar}{.42\textandmarginlen}{
\begin{exampaper}
\question Evaluate
\[
\int_{-1}^1 x^2\cosh{x} \dx
\]
\begin{shortchoices}%
\choicenumber $\frac{2\sinh{1}}{3}$\choicespace%
\choicenumber $-\frac{2\sinh{1}}{3}$\\
\choicenumber $\frac{2\cosh{1}}{3}$\choicespace%
\choicenumber $-\frac{2\cosh{1}}{3}$%
\end{shortchoices}
\end{exampaper}
}
\begin{lstlisting}
\begin{exampaper}
\question Evaluate
\[\int_{-1}^1 x^2\cosh{x} \dx\]
\begin{shortchoices}%
\choicenumber $\frac{2\sinh{1}}{3}$\choicespace \choicenumber $-\frac{2\sinh{1}}{3}\\ \choicenumber $\frac{2\cosh{1}}{3}$\choicespace \choicenumber $-\frac{2\cosh{1}}{3}$%
\end{shortchoices}
\end{exampaper}
\end{lstlisting}
\end{miniexammar}
Here, \raul{} uses the comment sign \verb=%= to eliminate the line break character, which would be treated as a white space by \LaTeX{} if not removed.

For longchoices environment, the usage is similar, except that \raul{} doesn't need to care about the line breaks, as one choice itself begins a new line at its end.
\begin{miniexammar}{.42\textandmarginlen}{
\begin{exampaper}
\question Which of the following functions satisfy Laplace's Equation
\[\frpdt[^2 f]{x^2} + \frpdt[^2 f]{y^2} = 0\]
\begin{longchoices}%
\choicenumber $f(x,y):=x^3y-xy^3$\choicespace%
\choicenumber $f(x,y):=\log{(4x^2+4y^2)}$\choicespace%
\choicenumber $f(x,y):=3 x^4 y^5 - 2 x^2 y^3$\choicespace%
\choicenumber $f(x,y):=\cos{(2x^2-2y^2)}$%
\end{longchoices}
\end{exampaper}
}
\begin{lstlisting}
\begin{exampaper}
\question Which of the following functions satisfy Laplace's Equation
\[\frpdt[^2 f]{x^2} + \frpdt[^2 f]{y^2} = 0\]
\begin{longchoices}%
\choicenumber $f(x,y):=x^3y-xy^3$\choicespace%
\choicenumber $f(x,y):=\log{(4x^2+4y^2)}$\choicespace%
\choicenumber $f(x,y):=3 x^4 y^5 - 2 x^2 y^3$\choicespace%
\choicenumber $f(x,y):=\cos{(2x^2-2y^2)}$%
\end{longchoices}
\end{exampaper}
\end{lstlisting}
\end{miniexammar}

\subsection{Space And Rules}
It is conventional to leave a vertical space after a comprehensive question is given. To add a certain amount of vertical space, \raul{} uses command \verb=\vspace=.
\begin{miniexammar}{.5\textandmarginlen}{
\begin{exampaper}
\question Find ...
\vspace{.8cm}
\question Prove ...
\end{exampaper}
}
\begin{lstlisting}
\question Find ...
\vspace{.8cm}
\question Prove ...
\end{lstlisting}
\end{miniexammar}

He is quite satisfied with this result, though he wonders how to evenly distribute questions on a page. The command \verb=\stretch=, when used in \verb=\vspace=, gives a space with respect to the number given to it. For example, if \raul{} wants to place three questions evenly on a page, he writes:
\begin{miniexammar}{.5\textandmarginlen}{
\begin{exampaper}
\question Find ...
\vspace{.5cm}
\question Prove ...
\vspace{.5cm}
\question Evaluate ...
\end{exampaper}
}
\begin{lstlisting}
\question Find ...
\vspace{\stretch{1}}
\question Prove ...
\vspace{\stretch{1}}
\question Evaluate ...
\end{lstlisting}
\end{miniexammar}
The number can be changed for a relatively larger space. For example, if \raul{} writes
\begin{lstlisting}
\question Find ...
\vspace{\stretch{2}}
\question Prove ...
\vspace{\stretch{1}}
\question Evaluate ...
\end{lstlisting}
, then the empty space left will be evenly divided into $2+1=3$ pieces. Two of them are given to the place where \verb=\stretch{2}= is given, and the other one is given to the place where \verb=\stretch{1}= is given.

To create questions that require a student to fill in the blanks, \raul{} has to draw the blanks, that is, to draw a line under each of the blanks. In \LaTeX{}, a horizontal line can be drawn by the command
\begin{verbatim}
\rule[lift]{width}{height}
\end{verbatim}
, where lift is with respect to the baseline.

To simplify the work, xjtluexam provides the command \verb=\blankline= to draw a line with a given width (default 1cm).
\begin{miniexammar}{.5\textandmarginlen}{
\begin{exampaper}
\question $(3\vec{i}-9\vec{j}) \times (2\vec{i}+1\vec{j}) = $ \blankline
\question $\frdt[\cos^2 x \sin x]{x} = $ \blankline[1.5cm]
\end{exampaper}
}
\begin{lstlisting}
\begin{exampaper}
\question $(3\vec{i}-9\vec{j}) \times (2\vec{i}+1\vec{j}) = $ \blankline
\question $\frdt[\cos^2 x \sin x]{x} = $ \blankline[1.5cm]
\end{exampaper}
\end{lstlisting}
\end{miniexammar}

Now \raul{} has everything he needs to know, and an exam paper will be ready soon...