\section{Tutorial 2: Delilah and Complex Math Formulae}
Delilah is working on a part of a material about linear algebra. As her work proceeds, she will obtain the ability to deal with complex mathematical formulae in \LaTeX{}, especially those methods provided by the ams packages loaded in xjtlumath.

\subsection{Multiple Lined Formulae}
A system of linear equations is a fundamental part of linear algebra. When Delilah tries to type a group of equations, she encounters a problem. In the predefined \verb=\[\]= and environment equation, she finds no option to start a new line. Even the line-break options of \LaTeX{} like \verb=\\= and \verb=\newline= do not work there. Of course the equations should not be put in one line, so what should she do now? Later she learns that the environment \emph{aligned} is designed to allow a system of equations to be aligned in multiple lines:

\begin{miniexammar}{.4\textandmarginlen}{
\[
\begin{aligned}
x+y &= 1\\
x-y &= 2
\end{aligned}
\]
}
\begin{lstlisting}
\[
\begin{aligned}
x+y &= 1\\
x-y &= 2
\end{aligned}
\]
\end{lstlisting}
\end{miniexammar}
Here, the ampersand sign \verb=&= is used before the symbols according to which the equations are to be aligned. The line break sign \verb=\\= starts a new line of equation. Note that other line-break operations cannot be used here.

Delilah likes the result, but she feels that the equations are too lonely. She thinks that adding a large curly bracket for them will comfort them. \LaTeX{} supports a syntax to put things before and after a group of things.

\begin{miniexammar}{.4\textandmarginlen}{
\[
\left\{
\begin{aligned}
x+y &= 1\\
x-y &= 2
\end{aligned}
\right.
\]
}
\begin{lstlisting}
\[
\left\{
\begin{aligned}
x+y &= 1\\
x-y &= 2
\end{aligned}
\right.
\]
\end{lstlisting}
\end{miniexammar}
The \verb=\left= command defines what is to be put on the left, and the \verb=\right= command defines what is to be put on the right. Delilah does not want to put anything on the right, so she writes \verb=.= for nothing.

For a group of equations that requires no alignment, or for a single equation that is too long to fit in one line, the environment \emph{gathered} that does no alignment is a better choice:

\begin{miniexammar}{.45\textandmarginlen}{
\[
\begin{gathered}
\cos {z} = 1 - \frac{z^2}{2!} + \frac{z^4}{4!} - \frac{z^6}{6!} + \cdots \\
= \sum_{n=0}^\infty {\frac{(-1)^n z^{2n}}{(2n!)}}
\end{gathered}
\]
}
\begin{lstlisting}
\[
\begin{gathered}
\cos {z} = 1 - \frac{z^2}{2!} + \frac{z^4}{4!} - \cdots \\
= \sum_{n=0}^\infty {\frac{(-1)^n z^{2n}}{(2n!)}}
\end{gathered}
\]
\end{lstlisting}
\end{miniexammar}

After writing several groups of equations, Delilah wants to reference one of them. She uses the equation environment instead of \verb=\[\]=, but finds out that the equations are numbered as a whole.

\begin{miniexammar}{.4\textandmarginlen}{
\begin{equation}
\begin{aligned}
x+y &= 1\\
x-y &= 2
\end{aligned}
\end{equation}
}
\begin{lstlisting}
\begin{equation}
\begin{aligned}
x+y &= 1\\
x-y &= 2
\end{aligned}
\end{equation}
\end{lstlisting}
\end{miniexammar}
So it is difficult for her to reference a single equation in a group. amsmath provides the environment align for this purpose:

\begin{miniexammar}{.4\textandmarginlen}{
\begin{align}
x+y &= 1\\
x-y &= 2
\end{align}
}
\begin{lstlisting}
\begin{align}
x+y &= 1\\
x-y &= 2
\end{align}
\end{lstlisting}
\end{miniexammar}

If she doesn't want to number a single equation, she needs to append \verb=\nonumber= at the end of that line.
\begin{miniexammar}{.4\textandmarginlen}{
\begin{align}
x+y &= 1\\
z&=10 \nonumber\\
x-y &= 2
\end{align}
}
\begin{lstlisting}
\begin{align}
x+y &= 1\\
z &= 10 \nonumber\\
x-y &= 2
\end{align}
\end{lstlisting}
\end{miniexammar}

Without the ``ed'' suffix, \emph{gather} is also a standalone environment that does what gathered do. But there is a major difference between the normal version and ``ed''ed version. Delilah finds it impossible to put the bracket again before a align or gather, because they don't need to be surrounded by mathematical environments. Also, their width are fixed to be the width of texts, while their ``ed''ed versions can be of any width.

As the same as the equation environment, their starred versions give no number by default.
\begin{miniexammar}{.4\textandmarginlen}{
\begin{align*}
x+y &= 1\\
x-y &= 2
\end{align*}
}
\begin{lstlisting}
\begin{align*}
x+y &= 1\\
x-y &= 2
\end{align*}
\end{lstlisting}
\end{miniexammar}

Delilah is able to put multiple groups of equations in one align, just by adding ampersands between the groups.
\begin{miniexammar}{.4\textandmarginlen}{
\begin{align*}
x+y &= 1  & a+b &= 3\\
x-y &= 2  & a-b &= 4
\end{align*}
}
\begin{lstlisting}
\begin{align*}
x+y &= 1  & a+b &= 3\\
x-y &= 2  & a-b &= 4
\end{align*}
\end{lstlisting}
\end{miniexammar}
The space between the groups is adjusted automatically by align.

\subsection{Matrices}
Matrices are vital to linear algebra, as they represent linear mappings from a vector space to another in specific bases. Also, the coefficient matrix and the augmented matrix are convenient in operating linear equations.

amsmath provides several environments for typing matrices.
\begin{miniexammar}{.5\textandmarginlen}{
\[
\begin{bmatrix}
1&2&3&4\\
5&6&7&8\\
9&10&11&12\\
13&14&15&16
\end{bmatrix}
\]
}
\begin{lstlisting}
\[
\begin{bmatrix}
1&2&3&4\\
5&6&7&8\\
9&10&11&12\\
13&14&15&16
\end{bmatrix}
\]
\end{lstlisting}
\end{miniexammar}
The environments pmatrix, Bmatrix, vmatrix, and Vmatrix produce delimiters of \verb=()=, \verb={}=, \verb=||=, and \verb=|| ||=, respectively.

To use matrices in inline mode, Delilah uses the environment smallmatrix, which has no p,b,B,v,V versions in amsmath, as it is the author's responsibility to decide the delimiters.
\begin{miniexammar}{.4\textandmarginlen}{
The matrix $\left(\begin{smallmatrix} a&b\\c&d \end{smallmatrix}\right)$ is so small and cute!
}
\begin{lstlisting}
The matrix $\left(\begin{smallmatrix} a&b\\c&d \end{smallmatrix}\right)$ is so small and cute!
\end{lstlisting}
\end{miniexammar}

When Delilah tries to put fractions inside a matrix, she finds something annoying.
\begin{miniexammar}{.3\textandmarginlen}{
\[
\begin{bmatrix}
1&\frac{1}{2}&\frac{1}{3}\\
1&\frac{1}{4}&\frac{1}{5}\\
\end{bmatrix}
\]
}
\begin{lstlisting}
\[
\begin{bmatrix}
1&\frac{1}{2}&\frac{1}{3}\\
1&\frac{1}{4}&\frac{1}{5}\\
\end{bmatrix}
\]
\end{lstlisting}
\end{miniexammar}
The fractions above and below are so close that they touch each other! This is not what Delilah wants and she is surprised that \LaTeX{} doesn't detect this and do something. Fortunately, in amsmath environments, an optional argument is allowed to be passed to \verb=\\= to define the actual vertical space between lines. For fractions, 2ex is a good option. Also, the fractions are in inline mode. The \verb=\dfrac= command gives displayed fractions.
\begin{miniexammar}{.3\textandmarginlen}{
\[
\begin{bmatrix}
1&\dfrac{1}{2}&\dfrac{1}{3}\\[2ex]
1&\dfrac{1}{4}&\dfrac{1}{5}
\end{bmatrix}
\]
}
\begin{lstlisting}
\[
\begin{bmatrix}
1&\dfrac{1}{2}&\dfrac{1}{3}\\[2ex]
1&\dfrac{1}{4}&\dfrac{1}{5}
\end{bmatrix}
\]
\end{lstlisting}
\end{miniexammar}

Sometimes a matrix is too large to be displayed fully. At these times, the use of ellipses (plural of ellipsis, not ellipse) is important. When Delilah writes the inverse of a matrix, she uses ellipses.
\begin{miniexammar}{.5\textandmarginlen}{
\[
A^{-1} = \frac{1}{\det A}
\begin{bmatrix}
C_{11} & C_{21} & \cdots & C_{n1} \\
C_{12} & C_{22} & \cdots & C_{n2} \\
\vdots & \vdots & \ddots & \vdots \\
C_{n2} & C_{n2} & \cdots & C_{nn} \\
\end{bmatrix}
\]
}
\begin{lstlisting}
\[
A^{-1} = \frac{1}{\det A}
\begin{bmatrix}
C_{11} & C_{21} & \cdots & C_{n1} \\
C_{12} & C_{22} & \cdots & C_{n2} \\
\vdots & C\vdots & \ddots & \vdots \\
C_{n2} & C_{n2} & \cdots & C_{nn} \\
\end{bmatrix}
\]
\end{lstlisting}
\end{miniexammar}

\subsection{Texts and Operator Names}
To put text inside math environments, Delilah uses the \verb=\text= command provided by amsmath.
\begin{miniexammar}{.57\textandmarginlen}{
\begin{definition}[Null Space]
The null space of an $m \times n$ matrix $A$, written as $\Nul A$, is the set of all solutions
of the homogeneous equation $A\vec{x} = \vec{0}$. In set notation,
\[
\Nul A = \{\vec{x}:\vec{x} \text{ is in } \mathbb{R}^n \text{ and } A\vec{x} = \vec{0} \}
\]
\end{definition}
}
\begin{lstlisting}
\begin{definition}[Null Space]
The null space of an $m \times n$ matrix $A$, written as $\Nul A$, is the set of all solutions
of the homogeneous equation $A\vec{x} = \vec{0}$. In set notation,
\[
\Nul A = \{\vec{x}:\vec{x} \text{ is in } \mathbb{R}^n \text{ and } A\vec{x} = \vec{0} \}
\]
\end{definition}
\end{lstlisting}
\end{miniexammar}

The commands like \verb=\Nul=, \verb=\sin=, ... are math operators. Part of predefined math operators in \LaTeX{} are shown in Table \ref{tab:predefmathop}.
\begin{table}[hbpt]
\begin{center}
\small
\begin{tabular}{cl|cl|cl}
Result & Command & Result & Command & Result & Command \\
\hline
arccos & \verb=\arccos= & arcsin & \verb=\arcsin= & arctan & \verb=\arctan= \\
cos & \verb=\cos= & sin & \verb=\sin= & tan & \verb=\tan= \\
cot & \verb=\cot= & sec & \verb=\sec= & csc & \verb=\csc= \\
cosh & \verb=\cosh= & sinh & \verb=\sinh= & tanh & \verb=\tanh= \\
lim & \verb=\lim= & lim inf & \verb=\liminf= & lim sup & \verb=\limsup= \\
ln & \verb=\ln= & log & \verb=\log= & lg & \verb=\lg= \\
max & \verb=\max= & min & \verb=\min= & sup & \verb=\sup= \\
inf & \verb=\inf= &  &  &  &  \\
ker & \verb=\ker= & det & \verb=\det= & exp & \verb=\exp= 
\end{tabular}
\end{center}
\caption{Some Predefined Math Operators}
\label{tab:predefmathop}
\end{table}

In fact, operator \verb=\Nul= and \verb=\Span= are defined by xjtlumath as in the forms in the year 1 linear algebra textbook of XJTLU. Also, xjtlumath changes the default \verb=\vec= command in \LaTeX{} so that vectors appear in bold form rather than with a arrow above them.

Some operators, like \verb=\lim=, are designed to support taking limits. That is, in displayed mode, when one tries to give one of such operators a subscript using \verb=_=, the subscript will appear at the bottom of the operator.
\begin{parexammar}{.4\textandmarginlen}{
\[
\lim_{x\to 0} f(x)
\]
}
\begin{lstlisting}
\[
\lim_{x\to 0} f(x)
\]
\end{lstlisting}
\end{parexammar}

Delilah is able to explicitly control the limit style by using \verb=\limits= and \verb=\nolimits=. Note that these two commands can only be used after a operation that supports taking limits.
\begin{parexammar}{.4\textandmarginlen}{
$\lim\limits_{x \to 0}f(x)$
\[
\lim\nolimits_{x\to 0} f(x)
\]
}
\begin{lstlisting}
$\lim\limits_{x \to 0}f(x)$
\[
\lim\nolimits_{x\to 0} f(x)
\]
\end{lstlisting}
\end{parexammar}

\subsection{Delimiters}
Delilah already knows how to type basic delimiters. For parentheses and square brackets, plain text will do; and since the curly brackets are reserved by \LaTeX{}, Delilah needs to add a backslash before each of them.
\begin{parexammar}{.45\textandmarginlen}{
The range of a function may be expressed explicitly by its domain and itself: the range of $f: X \to Y$ is $f(X)$.

In some context the arguments of a function are enclosed by square brackets: $f[x]$.

Curly brackets are often used to show a set: $S := \{2,4,\cdots\}$.
}
\begin{lstlisting}
The range of a function may be expressed explicitly by its domain and itself: the range of $f: X \to Y$ is $f(X)$.

In some context the arguments of a function are enclosed by square brackets: $f[x]$.

Curly brackets are often used to show a set: $S := \{2,4,\cdots\}$.
\end{lstlisting}
\end{parexammar}

Yet in some conditions where the expression being enclosed has a different height, the result becomes unsatisfactory. \LaTeX{} provides a mechanism to enable the user to automatically or manually adjust the size of delimiters.

Adding \verb=\left= and \verb=\right= around a pair of delimiters automatically adjust their size to match the expression being enclosed. However, sometimes we want the delimiters to be a bit bigger or smaller, and that's when we need to manually adjust the size.
\begin{parexammar}{.5\textandmarginlen}{
\[
\left( \frac{1}{2} \right) 
\Bigg( \bigg( \Big( \big( x\big) \Big) \bigg) \Bigg)
\]
}
\begin{lstlisting}
\[
\left(\frac{1}{2}\right) \Bigg( \bigg( \Big( \big( x\big) \Big) \bigg) \Bigg)
\]
\end{lstlisting}
\end{parexammar}

Delilah once mismatched a pair of delimiters, but she found that the height of the two symbols are still the same. She learns in addition that the dot `.' can be used to inform \LaTeX{} to insert nothing.
\begin{parexammar}{.5\textandmarginlen}{
A fraction enclosed left by a parenthesis, and right by a curly bracket.
\[
\left(\frac{a}{b}\right\}
\]
A system of equations
\[
\left\{
\begin{aligned}
x+y&=1\\
x-y&=2
\end{aligned}
\right.
\]
}
\begin{lstlisting}
A fraction enclosed left by a parenthesis, and right by a curly bracket.
\[
\left(\frac{a}{b}\right\}
\]
A system of equations
\[
\left\{
\begin{aligned}
x+y&=1\\
x-y&=2
\end{aligned}
\right.
\]
\end{lstlisting}
\end{parexammar}

Now Delilah knows how to properly handle delimiters, but the repetitive \verb=\left= and \verb=\right= really makes her sick. xjtlumath simplifies the work by providing these predefined groups of delimiters. The following examples show the delimiter groups defined by xjtlumath, as well as the explicit control of the size.
\begin{parexammar}{.5\textandmarginlen}{
These delimiters are defined by xjtlumath.
\[
\rbra{x},\ \sbra{x},\ \cbra{x},\ \abs{x},\ \floor{x},\ \ceiling{x}
\]
One can also change the delimiter resizer.
\[
\abs[\bigg]{x}
\]
}
\begin{lstlisting}
These delimiters are defined by xjtlumath.
\[
\rbra{x},\ \sbra{x},\ \cbra{x},\ \abs{x},\ \floor{x},\ \ceiling{x}
\]
One can also change the delimiter resizer.
\[
\abs[\bigg]{x}
\]
\end{lstlisting}
\end{parexammar}

\subsection{Symbols}
The standard \LaTeX{} font in math environments is neat and clean. Yet in some special occasions Delilah would like to change the font of certain symbols. For example, to represent some conventional sets, she uses the blackboard font.
\begin{parexammar}{.4\textandmarginlen}{
\[
\mathbb{R}\quad \mathbb{N}\quad \mathbb{Q}\quad \mathbb{Z}
\]
}
\begin{lstlisting}
\[
\mathbb{R}\quad \mathbb{N}\quad \mathbb{Q}\quad \mathbb{Z}
\]
\end{lstlisting}
\end{parexammar}

Writing \verb=\mathbb= every time is somehow irritating. For this purpose, xjtlumath defines shorthands for them.
\begin{parexammar}{.45\textandmarginlen}{
\[
\setr \quad \setq \quad \setz \quad \setn \quad \setnp
\]
}
\begin{lstlisting}
\[
\setr \quad \setq \quad \setz \quad \setn \quad \setnp
\]
\end{lstlisting}
\end{parexammar}

Other font controlling methods are like what we have talked about in subsection \ref{subsec:fonts}. For example, \verb=\mathrm= gives font in \textrm{Roman} family, and \verb=\mathbf= gives font in \textbf{bold} series.
\begin{parexammar}{.45\textandmarginlen}{
\[
\mathrm{Like normal text} \quad \mathbf{bold}
\]
}
\begin{lstlisting}
\[
\mathrm{Like normal text} \quad \mathbf{bold}
\]
\end{lstlisting}
\end{parexammar}