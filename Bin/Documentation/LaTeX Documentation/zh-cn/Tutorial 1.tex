\part{教程}
\pagenumbering{arabic}
\section{教程 1: Ashley的第一份资料}
Ashley 最近加入了资料部门。 现在他被指派写一部分涵盖基本微积分的资料。 他非常兴奋,因为这将是他的第一个资料,也是他对 \LaTeX 的第一次尝试。 首先,他需要知道如何在他的 PC 上使用 \LaTeX{} 和别的宏包。 在本教程中,我们将跟随 Ashley,看看他在资料准备中学到了什么。

\subsection{安装和配置}
要使用 \LaTeX{},Ashley 需要在他的计算机上安装一个 \TeX{} 发行版(可能还有一个编辑器)。 \LaTeX{} 官方网站上列出了几种流行的发行版:\url{https://www.latex-project.org/get/}。 这些发行版的安装和配置非常简单,Ashley 在几分钟内完成了它们。

然后,Ashley 想要一个编辑器来编写 \LaTeX{} 文档。 他了解到 \TeX studio 是一个很好的工作室,所以他下载并安装了它。

Ashley 现在需要做的是获取该宏包的副本并安装它。 Ashley 必须转到 GitHub 存储库 \gitrepo{} 以获取副本。 他导航到存储库并转到最新版本 \url{https://github.com/Little-He-Guan/A-XJTLU-Math-Club-LaTeX-Template/releases},从那里他下载 xjtlumath.zip 在最新版本下。

现在他拿到了包裹并解压了它。 他的 PC 上的操作系统是 Windows,因此他可以简单地将压缩包内容复制到准备他的文档的路径下并运行 install.bat。

对于其他操作系统,复制内容后,必须进入命令行,导航到工作路径,然后键入
\begin{verbatim}
latex xjtlumath.ins
\end{verbatim}
来抽取宏包。 

\subsection{基本书写}
Ashley,迫不及待地等待开始他的第一项工作,单击打开 Templates 文件夹并导航到 material-book 文件夹。 他打开encapsulation.tex,发现它是这样的:
\begin{lstlisting}
... Some code ...

\input{chapters.tex}

... Some code ...
\end{lstlisting}

正如文件名所暗示的,这是用于制作将作为书籍出版的资料。 一本书包含一些章节,在资料写作中,每一章都是资料主题的一个特定部分。 例如,Ashley 的工作是微积分资料的一部分,因此他需要为他的工作开始一章。 他打开chapter.tex 并发现它是一个空白文档。 然后他添加以下命令开始他的第一章并在 encapsulation.tex 上运行 \LaTeX{} 以检查输出(左侧是他的命令的结果,出现在不同的字体系列中,右侧是 是他的命令,背景为灰色):

\begin{miniexammar}{.45\textandmarginlen}{\fakesectioningdef{1}{Key points in calculus}}
\begin{lstlisting}
\chapter{Key points in calculus}
\end{lstlisting}
\end{miniexammar}
\verb=\chapter{...}= 是一个 \LaTeX{} \emph{command},它以 \verb=\=% 开头
. 用大括号 \verb={}= 包裹的单词形成命令的 \emph{参数}。 在这个地方,参数是章节的标题。 Ashley 注意到 \LaTeX{} 会自动放大标题字体并使它们加粗。 与许多其他排版软件不同的是,用\LaTeX{} 只需要知道要做什么的逻辑思路(例如在某个地方会有一个名为 xxx 的章节),它会为作者控制外观。

Ashley对这个结果很满意。 然后他键入一些段落。 在 \LaTeX{} 中,段落由一个空行分隔。

\begin{miniexammar}{.5\textandmarginlen}{
\fakesectioningdef{1}{Key points in calculus}%
What does Ashley write in these paragraphs? Well, in fact, I don't know. You may find him and ask him yourself.
		
\hspace{1.5em}But, wait a minute, how do I find Ashley when he doesn't really exist? Well, this is a good question.}
\begin{lstlisting}
\chapter{Key points in calculus}
What does Ashley write in these paragraphs? Well, in fact, I don't know. You may find him and ask him yourself.
		
But, wait a minute, how do I find Ashley when he doesn't really exist? Well, this is a good question.
\end{lstlisting}
\end{miniexammar}
Ashley 注意到段落由 \LaTeX{} 自动缩进。 然而,他也注意到该章正下方的段落没有缩进(在中文环境中有所不同)。

除了章节之外,\LaTeX{} 还提供了以下分节命令:
\begin{itemize}
\item section
\item subsection
\item subsubsection
\item paragraph
\item subparagraph
\end{itemize}
您可能会注意到此处包含该paragraph。 事实上,分节命令 \verb=\paragraph= 像其他分节命令一样为段落生成标题。

当 Ashley 转向整个输出时,他看到他的章节出现在目录中:

\begin{miniexammar}{.7\textandmarginlen}{
\faketoc
\fakecontentsline{1}{\textbf{Key points in calculus}}{1}
}
\begin{lstlisting}
\chapter{Key points in calculus}
\end{lstlisting}
\end{miniexammar}
\LaTeX{} 为所有分节命令生成目录\footnote{其深度低于目录 (toc) 深度。 toc的实际生成过程有点复杂,这里就不赘述了}。 出于这个原因,Ashley 必须在 encapsulation.tex 上运行 \LaTeX{} 两次以获得正确的目录。

当 Ashley 为某个部分写了很长的标题时,目录变得非常难看。 为了解决这个问题,Ashley 可以指定目录中使用的节的标题的简短形式,如下所示:
\begin{miniexammar}{.7\textandmarginlen}{
\faketoc
\fakecontentsline{1}{\textbf{A Short Name}}{1}
}
\begin{lstlisting}
\chapter[A Short Name]{A very very long Caption}
\end{lstlisting}
\end{miniexammar}

\subsection{调整字体} \label{subsec:fonts}
到目前为止,Ashley 知道如何指示 \LaTeX{} 做一些基本的事情。 虽然他感觉精力充沛,正在全速写作,但很快就遇到了一些问题。 Ashley 想强调一些关键字,例如“limit”。 他后来了解到命令 \verb=\emph= 指示 \LaTeX{} 强调传递给它的文本,如下所示。

\begin{parexammar}{.38\textandmarginlen}{
Calculus is the study of \emph{limits}.
}
\begin{lstlisting}
Calculus is the study of \emph{limits}.
\end{lstlisting}
\end{parexammar}

这东西挺\LaTeX,因为Ashley 只告诉\LaTeX{} 强调它,而无法控制\LaTeX{} 如何做到这一点。 虽然大多数时候这就足够了,但Ashley想要更多。 他想知道如何明确控制文本的外观,因为 \LaTeX{} 无法满足所有情况下的所有需求。 \LaTeX{} 确实提供了对字体的某些默认操作,Ashley 可以使用它们来控制文本的大小、系列和样式。
\begin{parexammar}{.45\textandmarginlen}{
Ashley can tell \LaTeX{} to adjust the font size like:
{\tiny very very small} {\scriptsize the size of scripts} {\footnotesize the size of foot notes} {\small small font} {\normalsize just being normal} {\large a bit bigger} {\Large large text} {\LARGE very big} {\huge huge} {\Huge damn huge}
}
\begin{lstlisting}
Ashley can tell \LaTeX{} to adjust the font size like:
{\tiny very very small} {\scriptsize the size of scripts} {\footnotesize the size of foot notes} {\small small font} {\normalsize just being normal} {\large a bit bigger} {\Large large text} {\LARGE very big} {\huge huge} {\HUGE damn huge}
\end{lstlisting}
\end{parexammar}
这一次 Ashley 看到了与命令 \verb=\emph{}= 不同的东西。 此处的文本在大括号内,命令与文本一起在括号内给出。 被一对大括号括起来的东西被称为在 \emph{组} 内。 在组内调用的命令会影响整个组。

除了大小,Ashley 还能够控制字体的样式和系列。 如表 \ref{tab:stdfontcmds} 所示,\LaTeX{} 提供了以下命令来控制字体样式和系列:
\begin{table}[htbp]
\begin{tabular}{lll}
Command & Used in a group & Action\\
\hline
\verb=\textrm{...}= & \verb={\rmfamily...}= & {Text in \textrm{Roman} family} \\
\verb=\textsf{...}= & \verb={\sffamily...}= & {Text in \textsf{sans serif} family} \\
\verb=\texttt{...}= & \verb={\ttfamily...}= & {Text in \texttt{typewriter} family} \vspace{.15cm}\\
\verb=\textmd{...}= & \verb={\mdseries...}= & {Text in \textmd{medium} series} \\
\verb=\textbf{...}= & \verb={\bfseries...}= & {Text in \textbf{bold} series} \vspace{.15cm}\\
\verb=\textup{...}= & \verb={\upshape...}= & {Text in \textup{upright} shape} \\
\verb=\textit{...}= & \verb={\itshape...}= & {Text in \textit{italic} shape} \\
\verb=\textsl{...}= & \verb={\slshape...}= & {Text in \textsl{slanted} shape} \\
\verb=\textsc{...}= & \verb={\scshape...}= & {Text in \textsc{small caps} shape} \vspace{.15cm}\\
\verb=\emph{...}= & \verb={\em...}= & {Text \emph{emphasized}}\vspace{.15cm}\\
\verb=\textnormal{...}= & \verb={\normalfont...}= & {Text in default font}
\end{tabular}
\caption{标准字体更改命令和声明}
\label{tab:stdfontcmds}
\end{table}

这些命令中的许多命令都提供了组内使用版本和普通版本。 Ashley 可以根据自己的需要选择使用哪个版本。

Ashley 还想学习如何更改字体颜色。 他很惊讶这个包文档没有提供这样的描述。 与包作者联系后得知,由于资料是黑白打印的,所以没有涉及这个主题,因此更改颜色几乎没有用。

\subsection{键入数学公式}
这是 Ashley 工作中最激动人心的部分 --- 输入公式!尽管他之前在 \LaTeX{} 方面的经验很少,但他已经了解到 \LaTeX{} 可以生成高质量的数学公式,正如他之前在 Math Stack Exchange (\url{https://math.stackexchange .com/})和知乎(\url{https://www.zhihu.com/})。

由于他之前在这些网站上的经验,他对如何在 \LaTeX{} 中编写公式有所了解。

通常,\LaTeX{} 中的公式分为两种类型:\emph{inline} 和\emph{displayed}。前一种类型的公式被一对美元符号包围:\verb=$...$=,而后一种类型的公式被一对双美元符号包围:\verb=$$...$$ =。这些分隔符是原始的 \TeX{} 分隔符。 \LaTeX{} 分别为内联和显示数学提供了两对分隔符:\verb=\(...\)= 和 \verb=\[...\]=。事实上,应该避免使用 \TeX{} 简写 \verb=$$...$$= 来显示数学,因为它可能会导致 \LaTeX{} 中出现奇怪的问题。

顾名思义,inline公式位于一行文本中,而displayed公式则显示在正文之外。

\begin{miniexammar}{.55\textandmarginlen}%
{
The derivative of a function $f$ can be written as $f'$, or as
\[
\frdt{x}
\]
}
\begin{lstlisting}
The derivative of a function $f$ can be written as $f'$, or as
\[
\frdt{x}
\]
\end{lstlisting}
\end{miniexammar}
在这里,Ashley 使用了 xjtlumath 提供的命令:\verb=\frdt=。 此命令有两个参数,其中第一个参数是可选的,默认值为 $f$。 它们分别代表函数和变量。 当Ashley想要将$g$对$x$的导数排版时,他会写
\verb=\frdt[g]{x}=,给出 $\frdt[g]{x}$。 可选参数被包含在两个方括号内传递给命令。 像 Ashley 一样小心,您可能会注意到,在 inline 模式下,公式会在一定程度上缩小,以便它可以放在一行中。

Ashley对渲染效果相当满意。 然后他写了一些方程式和文本。 但是当他想引用一个方程时,问题就出现了。 他如何处理他想要引用的等式? 在这里,Ashley被介绍了另一种给出显示数学的方法:环境equation。

\begin{miniexammar}{.5\textandmarginlen}%
{
The fundamental theorem of calculus can be expressed in the form of Equation \ref{eq:fundthmcal}
\begin{equation}
\int_a^b f(x) \dx = F(b) - F(a) \label{eq:fundthmcal}
\end{equation}
}
\begin{lstlisting}
The fundamental theorem of calculus can be expressed in the form of Equation \ref{eq:fundthmcal}
\begin{equation}
\int_a^b f(x) \dx = F(b) - F(a) \label{eq:fundthmcal}
\end{equation}
\end{lstlisting}
\end{miniexammar}

Ashley 通过他之前的知识理解了这个等式:下划线 (\verb=_=) 将下标 ($a$) 引入积分符号 $\int$ (\verb=\int=),而插入符号 (\verb=^=) 将上标 ($b$) 引入其中。 命令 \verb=\dx= 由 xjtlumath 提供。 如果Ashley直接输入dx,结果会是这样的$\int f(x) dx$,难看。 \verb=\dx= 优化结果。 注意\verb=\dx= 只能用来表示积分变量,因为它在它前面加了一点空格。 如果Ashley 需要使用其他变量,他需要使用命令\verb=\dd=。 xjtlumath 也为多重积分提供了相同的功能。

\begin{miniexammar}{.6\textandmarginlen}%
{
\[
\int f(t) \dd t,\quad
\iint f(x,y) \dxdy,\quad
\iiint f(x,y,z) \dxdydz
\]
\[
\iint f \drdt,\quad
\iiint f \dzdrdt,\quad
\iiint f \drdtdp
\]
}
\begin{lstlisting}
\[
\int f(t) \dd t,\quad
\iint f(x,y) \dxdy,\quad
\iiint f(x,y,z) \dxdydz
\]
\[
\iint f \drdt,\quad
\iiint f \dzdrdt,\quad
\iiint f \drdtdp
\]
\end{lstlisting}
\end{miniexammar}

然而,Ashley 对 \LaTeX{} 中的交叉引用以及环境equation一无所知。让我们向Ashley解释它们。在\LaTeX{} 中,一个环境由\verb=\begin= 命令开始并以\verb=\end= 命令结束。环境的名称被传递给这对命令。环境equation给出了一个显示的数学方程,它是 \emph{被计数的}。equation末尾的 \verb=\label= 命令捕获计数器以及其他一些信息,例如其位置,并将其存储在由赋予 \verb=\label= 的名称表示的标签中。要使用标签,Ashley 需要 \verb=\ref= 命令,该命令打印计数器\footnote{并另外生成一个可点击的超链接,点击时导航到方程的位置,这是 hyperref ,此模板加载的包的效果。}。

equation 环境给出了一个计数器,而 \verb=\[\]= 没有。在后者使用标签会导致标签被定向到另一个计数器,因此仅当该事物被计数时才使用标签。

\subsection{空白管理}
\LaTeX{} 中的空间管理比仅仅输入空格要复杂一些。 Ashley是个细心的人,他很快发现句子后面的空格比单词之间的空格大一点(你可以放大.pdf文件看到这个)。 \LaTeX{} 决定一个空格作为句子末尾的空格,如果
\begin{enumerate}
\item 终止句子的标点符号后紧跟此空格,并且
\item 如果终止句子的标点符号是句号,则句号前的字母要是小写。
\end{enumerate}
大多数句子都按照上述规则结束,但也有一些例外。 例如,\LaTeX{} 可能将 Mr.~ 作为句子结尾的符号,从而产生错误的间距。 在这种情况下,Ashley 需要通过 \verb=~= 和 \verb=\@= 手动配置 \LaTeX{}。

\begin{parexammar}{.5\textandmarginlen}%
{
In another world, Mr.~Ashley was once loved by Miss Scarlett. In this world, Mr.~Ashley has a PC\@. He loves programming on his PC.
}
\begin{lstlisting}
In another world, Mr.~Ashley was once loved by Miss Scarlett. In this world, Mr.~Ashley has a PC\@. He loves programming on his PC.
\end{lstlisting}
\end{parexammar}
Ashley很高兴学会了如何管理空间。 然而很快他又发现了另一个问题。

\begin{parexammar}{.5\textandmarginlen}%
{
Some commands like \LaTeX seems to eat the space after it.
}
\begin{lstlisting}
Some commands like \LaTeX seems to eat the space after it.
\end{lstlisting}
\end{parexammar}
为了解决这个问题,Ashley 需要在命令的末尾添加一个空组(\verb={}=)。

至于数学公式,事情就变得不一样了。 \LaTeX{} 忽略数学模式下的所有空格,无论是内联的还是显示的。 要添加额外的空格,Ashley 必须使用表 \ref{tab:spacemath} 中显示的命令。
\begin{table}[htbp]
\begin{center}
\begin{tabular}{ll}
Command & Effect (approximately) \\
\hline
\verb=\,= & $\frac{3}{18}$ quad (\showwidth{.166666em}) \vspace{5pt}\\
\verb=\:= & $\frac{4}{18}$ quad (\showwidth{.222222em}) \vspace{5pt}\\
\verb=\;= & $\frac{5}{18}$ quad (\showwidth{.277777em}) \vspace{5pt}\\
\verb=\ = (\verb=\= followed by a space) & a space \vspace{5pt}\\
\verb=\quad= & Width of `M' in current font (\showwidth{1em}) \vspace{5pt}\\
\verb=\qquad= & 2 quad (\showwidth{2em})
\end{tabular}
\end{center}
\caption{空格在数学模式下}
\label{tab:spacemath}
\end{table}

\subsection{列表和其他环境}
现在Ashley已经学会了处理文本,他继续他的写作。很快,他不得不再次停止,因为他正在制定一份列表。 起初,他对列表进行了硬编码,如下所示:

\begin{parexammar}{.4\textandmarginlen}{
1. Something

2. Something

3. Something
}
\begin{lstlisting}
1. Something

2. Something

3. Something
\end{lstlisting}
\end{parexammar}
然而,这个解决方案看起来相当愚蠢。 另外,如果Ashley想要更改号码,他需要手动完成。 他想知道 \LaTeX{} 是否有一些更方便的方法来做到这一点。

幸运的是有。 \LaTeX{} 提供了几种处理列表的环境。 一个例子如下所示。

\begin{miniexammar}{.4\textandmarginlen}{
\begin{enumerate}
\item Something
\item Something
\item Something
\end{enumerate}
}
\begin{lstlisting}
\begin{enumerate}
\item Something
\item Something
\item Something
\end{enumerate}
\end{lstlisting}
\end{miniexammar}

现在Ashley几乎拥有他需要知道的一切。 他对自己写的东西很满意,感到很高兴。 现在对他来说重要的是他想以更花哨的方式展示定理、定义和其他东西,以便他的读者可以专注于这些。

为此,xjtlumath 提供了一些花哨的环境。

\begin{miniexammar}{.6\textandmarginlen}{
\begin{definition}[Absolute Convergence]
An infinite series $\infseries{n}{0}$ is said to be absolutely convergent iff
\[
\sum_{n=0}^\infty |a_n|
\]
converges
\end{definition}
}
\begin{lstlisting}
\begin{definition}[Absolute Convergence]
An infinite series $\infseries{n}{0}$ is said to be absolutely convergent iff
\[
\sum_{n=0}^\infty |a_n|
\]
converges
\end{definition}
\end{lstlisting}
\end{miniexammar}

xjtlumath 加载 amsthm,它提供了proof环境。 当 Ashley 使用这个环境写证明时,他发现在环境的末尾出现了一个 q.e.d.~符号。 
\begin{miniexammar}{.5\textandmarginlen}{
Now we prove the mean value theorem for definite integrals. That is, for a continuous function $f$ that is bounded on $[a,b]$, the definite integral $\int_a^b f(x) \dx = f(c)(b-a)$, where $c \in [a,b]$. 
\begin{proof}
Let $m,M$ be the infimum and supermum of $f([a,b])$, respectively.
Therefore, $m\le f \le M$, and
\[
\int_a^b m \dx \le \int_a^b f(x) \dx \le \int_a^b M \dx
\]
, which gives
\begin{align}
m(b-a) \le &\int_a^b f(x) \dx \le M(b-a) \nonumber\\
m \le &\frac{\int_a^b f(x) \dx}{b-a} \le M \label{eq:meanvalint}
\end{align}

Since that $f$ is continuous, it can reach every value between the infimum and supermum of its range. That is, $\exists c \in [a,b], f(c) = \int_a^b f(x) \dx$. Substitute $f(c)$ back to equation \ref{eq:meanvalint} gives what the theorem states.
\end{proof}
}
\begin{lstlisting}[basicstyle=\footnotesize]
Now we prove the mean value theorem for definite integrals. That is, for a continuous function $f$ that is bounded on $[a,b]$, the definite integral $\int_a^b f(x) \dx = f(c)(b-a)$, where $c \in [a,b]$. 
\begin{proof}
Let $m,M$ be the infimum and supermum of $f([a,b])$, respectively.
Therefore, $m\le f \le M$, and
\[
\int_a^b m \dx \le \int_a^b f(x) \dx \le \int_a^b M \dx
\]
, which gives
\begin{align}
m(b-a) \le &\int_a^b f(x) \dx \le M(b-a) \nonumber\\
m \le &\frac{\int_a^b f(x) \dx}{b-a} \le M \label{eq:meanvalint}
\end{align}

Since that $f$ is continuous, it can reach every value between the infimum and supermum of its range. That is, $\exists c \in [a,b], f(c) = \int_a^b f(x) \dx$. Substitute $f(c)$ back to equation \ref{eq:meanvalint} gives what the theorem states.
\end{proof}
\end{lstlisting}
\end{miniexammar}

Ashley 能够通过使用 proof 提供的 \verb=\qedhere= 命令来控制 q.e.d.~符号出现的位置。 如果这个命令在之前被给出,那么它就不会出现在最后。
\begin{miniexammar}{.5\textandmarginlen}{
\begin{proof}
Some words...
\[
a+b=c \qedhere
\]
Som words...
\end{proof}
}
\begin{lstlisting}
\begin{proof}
Some words...
\[
a+b=c \qedhere
\]
Som words...
\end{proof}
\end{lstlisting}
\end{miniexammar}

除了definition和proof,Ashley 还能够使用theorem, proposition, corollary, lemma, axiom, and example. 除了proof之外,这些环境拥有各自的计数器,Ashley 可以简单地使用标签来引用它们。

不过,Ashley的同事 超 抱怨到当他尝试写中文资料时,这些环境的标题还是英文。为了支持中文,xjtlumath给这些环境(不幸的是,在中文环境下,proof的标题也变成了中文,就没有英文版的了)增加了中文版本,只要在最后加一个c即可使用中文版本。

这些设施极大地帮助了Ashley的资料准备,他很快就会完成他的工作……