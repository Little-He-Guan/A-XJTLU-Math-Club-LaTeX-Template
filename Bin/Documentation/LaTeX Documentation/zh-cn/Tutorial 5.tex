\newcommand{\raul}{Ra\'ul}
\section{教程 5: \raul{} 排版试卷}
该部门的一项任务是阐明那些数学课程过去的试卷。 在本教程中,我们将看到 \raul{} 如何利用 xjtluexam 包来排版试卷。

\subsection{exampaper 环境}
要开始试卷,\raul{} 使用exampaper环境。 环境本身没有给出输出,但是提供了几个命令\footnote{其实命令不是这个环境\emph{提供}的。 它只是做了一些初始化工作,以便这些命令可以正常运行。}。

然后,\raul{} 输入 \verb=\question= 开始提问。
\begin{miniexammar}{.5\textandmarginlen}{
\begin{exampaper}
\question Justify whether $\infseries{n}{1}$ converges or not when $a_n \to 0$.
\question Evaluate
\[
\int x^5 e^{x^2} \dx
\]
\end{exampaper}
}
\begin{lstlisting}
\begin{exampaper}
\question Justify whether $\infseries{n}{1}$ converges or not if $a_n \to 0$.
\question Evaluate
\[
\int x^5 e^{x^2} \dx
\]
\end{exampaper}
\end{lstlisting}
\end{miniexammar}

问题计数器可以被正常引用。
\begin{miniexammar}{.5\textandmarginlen}{
\begin{exampaper}
\question Justify whether $\infseries{n}{1}$ converges or not when $a_n \to 0$.
\question \label{ques:examques} Evaluate
\[
\int x^5 e^{x^2} \dx
\]
\end{exampaper}
A solution to Question \ref{ques:examques} is ...
}
\begin{lstlisting}
\begin{exampaper}
\question Justify whether $\infseries{n}{1}$ converges or not when $a_n \to 0$.
\question \label{ques:examques} Evaluate
\[
\int x^5 e^{x^2} \dx
\]
\end{exampaper}
A solution to Question \ref{ques:examques} is ...
\end{lstlisting}
\end{miniexammar}

可以给出子问题,它们的编号是按字母顺序排列的。
\begin{miniexammar}{.5\textandmarginlen}{
\begin{exampaper}
\question Given the function $f=...$

\hspace{1.5em}(a) Evaluate $f(3)$

\hspace{1.5em}(b) Prove a proposition about $f$.
\end{exampaper}
}
\begin{lstlisting}
\begin{exampaper}
\question Given the function $f=...$
\subquestion Evaluate $f(3)$
\subquestion Prove a proposition about $f$.
\end{exampaper}
\end{lstlisting}
\end{miniexammar}

\subsection{选择题}
多项选择题 (MCQ) 经常出现在试卷中。 xjtluexam 包为他们设计了几个环境和命令。

为了引入 MCQ 的选择,\raul{} 开始了一个选择环境。 有两种选择环境:\emph{shortchoices} 和\emph{longchoices}。 顾名思义,它们分别用于引入短选项(多个选项在一行)和长选项(一个选项占一行)。

在任何一个环境中,\verb=\choicenumber= 命令给出了一个选择编号。 选项后的空格由命令 \verb=\choicespace= 给出。 \raul{} 不想在最后一个选项的末尾有一个空格,所以他只在选项之间使用命令。
\begin{miniexammar}{.55\textandmarginlen}{
\begin{exampaper}
\question Evaluate
\[
\int_{-1}^1 x^2\cosh{x} \dx
\]
\begin{shortchoices}
\choicenumber $\frac{2\sinh{1}}{3}$ \choicespace
\choicenumber $-\frac{2\sinh{1}}{3}$ \choicespace
\choicenumber $\frac{2\cosh{1}}{3}$ \choicespace
\choicenumber $-\frac{2\cosh{1}}{3}$ 
\end{shortchoices}
\end{exampaper}
}
\begin{lstlisting}
\begin{exampaper}
\question Evaluate
\[\int_{-1}^1 x^2\cosh{x} \dx\]
\begin{shortchoices}
\choicenumber $\frac{2\sinh{1}}{3}$ \choicespace \choicenumber $-\frac{2\sinh{1}}{3}$ \choicespace \choicenumber $\frac{2\cosh{1}}{3}$ \choicespace \choicenumber $-\frac{2\cosh{1}}{3}$ 
\end{shortchoices}
\end{exampaper}
\end{lstlisting}
\end{miniexammar}

如果 \raul{} 觉得空间太紧,他可以随意断行
\begin{miniexammar}{.42\textandmarginlen}{
\begin{exampaper}
\question Evaluate
\[
\int_{-1}^1 x^2\cosh{x} \dx
\]
\begin{shortchoices}%
\choicenumber $\frac{2\sinh{1}}{3}$\choicespace%
\choicenumber $-\frac{2\sinh{1}}{3}$\\
\choicenumber $\frac{2\cosh{1}}{3}$\choicespace%
\choicenumber $-\frac{2\cosh{1}}{3}$%
\end{shortchoices}
\end{exampaper}
}
\begin{lstlisting}
\begin{exampaper}
\question Evaluate
\[\int_{-1}^1 x^2\cosh{x} \dx\]
\begin{shortchoices}%
\choicenumber $\frac{2\sinh{1}}{3}$\choicespace \choicenumber $-\frac{2\sinh{1}}{3}\\ \choicenumber $\frac{2\cosh{1}}{3}$\choicespace \choicenumber $-\frac{2\cosh{1}}{3}$%
\end{shortchoices}
\end{exampaper}
\end{lstlisting}
\end{miniexammar}
此处,\raul{} 使用注释符号 \verb=%= 来消除换行符,如果不移除,该换行符将被 \LaTeX{} 视为一个空白符。

对于 longchoices 环境,用法类似,除了 \raul{} 不需要关心换行符,因为一个选择本身在其末尾开始新行。
\begin{miniexammar}{.42\textandmarginlen}{
\begin{exampaper}
\question Which of the following functions satisfy Laplace's Equation
\[\frpdt[^2 f]{x^2} + \frpdt[^2 f]{y^2} = 0\]
\begin{longchoices}%
\choicenumber $f(x,y):=x^3y-xy^3$\choicespace%
\choicenumber $f(x,y):=\log{(4x^2+4y^2)}$\choicespace%
\choicenumber $f(x,y):=3 x^4 y^5 - 2 x^2 y^3$\choicespace%
\choicenumber $f(x,y):=\cos{(2x^2-2y^2)}$%
\end{longchoices}
\end{exampaper}
}
\begin{lstlisting}
\begin{exampaper}
\question Which of the following functions satisfy Laplace's Equation
\[\frpdt[^2 f]{x^2} + \frpdt[^2 f]{y^2} = 0\]
\begin{longchoices}%
\choicenumber $f(x,y):=x^3y-xy^3$\choicespace%
\choicenumber $f(x,y):=\log{(4x^2+4y^2)}$\choicespace%
\choicenumber $f(x,y):=3 x^4 y^5 - 2 x^2 y^3$\choicespace%
\choicenumber $f(x,y):=\cos{(2x^2-2y^2)}$%
\end{longchoices}
\end{exampaper}
\end{lstlisting}
\end{miniexammar}

\subsection{空间和 Rules}
在给出一个综合问题后,通常会留下一个垂直空间。 要添加一定量的垂直空间,\raul{} 使用命令 \verb=\vspace=。
\begin{miniexammar}{.5\textandmarginlen}{
\begin{exampaper}
\question Find ...
\vspace{.8cm}
\question Prove ...
\end{exampaper}
}
\begin{lstlisting}
\question Find ...
\vspace{.8cm}
\question Prove ...
\end{lstlisting}
\end{miniexammar}

他对这个结果非常满意,尽管他想知道如何在页面上均匀分布问题。 命令\verb=\stretch=,当在\verb=\vspace= 中使用时,给出一个与给定数字相关的空间。 例如,如果 \raul{} 想在一页上均匀地放置三个问题,他写道:
\begin{miniexammar}{.5\textandmarginlen}{
\begin{exampaper}
\question Find ...
\vspace{.5cm}
\question Prove ...
\vspace{.5cm}
\question Evaluate ...
\end{exampaper}
}
\begin{lstlisting}
\question Find ...
\vspace{\stretch{1}}
\question Prove ...
\vspace{\stretch{1}}
\question Evaluate ...
\end{lstlisting}
\end{miniexammar}
对于相对较大的空间,可以更改数字。 例如,如果 \raul{} 写
\begin{lstlisting}
\question Find ...
\vspace{\stretch{2}}
\question Prove ...
\vspace{\stretch{1}}
\question Evaluate ...
\end{lstlisting}
,那么剩下的空间就会被平均分成$2+1=3$块。 其中两个给定\verb=\stretch{2}= 的地方,另一个给定\verb=\stretch{1}= 的地方。

要创建要求学生填写空白的问题,\raul{} 必须绘制空白,即在每个空白下方画一条线。 在\LaTeX{}中,可以通过以下命令绘制一条水平线
\begin{verbatim}
\rule[lift]{width}{height}
\end{verbatim}
,其中lift是相对于基线的。

为了简化工作,xjtluexam 提供了命令 \verb=\blankline= 来绘制具有给定宽度(默认为 1cm)的线。
\begin{miniexammar}{.5\textandmarginlen}{
\begin{exampaper}
\question $(3\vec{i}-9\vec{j}) \times (2\vec{i}+1\vec{j}) = $ \blankline
\question $\frdt[\cos^2 x \sin x]{x} = $ \blankline[1.5cm]
\end{exampaper}
}
\begin{lstlisting}
\begin{exampaper}
\question $(3\vec{i}-9\vec{j}) \times (2\vec{i}+1\vec{j}) = $ \blankline
\question $\frdt[\cos^2 x \sin x]{x} = $ \blankline[1.5cm]
\end{exampaper}
\end{lstlisting}
\end{miniexammar}

现在 \raul{} 拥有了他需要知道的一切,并且很快就会准备好试卷...