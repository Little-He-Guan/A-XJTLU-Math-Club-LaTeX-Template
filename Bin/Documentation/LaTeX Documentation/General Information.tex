\part{General Information}
\pagestyle{headings}
\section{What is this package?}
This \LaTeX{} package was originally intended to be used by the materials dept.~of XJTLU math club. Yet I found that my university lack \LaTeX{} templates for students, therefore I decided to extend it into a template for all users.

The package contains several useful commands and environments for mathematical documents, and redefines some existing styles so that they suit the university's style.

\section{History}
At the year 2020, I joined the materials dept.~of XJTLU math club. This department makes materials about math and distribute them to students to help them. At that time, the materials were being prepared with Microsoft Word and MathType. I saw that it was a good opportunity to enhance the materials with the power of \LaTeX, so I proposed to prepare the documents using \LaTeX.

The proposition was passed and our team started to transfer our working environment. During this process, I wrote a tiny package, a predecessor of this package, which was used in our works. Many people in our department knew nearly nothing about \LaTeX, and as I promised that the new preparation process would not be very hard, I managed to define the overall procedure, shared some necessary knowledge about \LaTeX, and taught them how to use the package.

I admit that I was a bit irresponsible at that time. Truly I described how the working procedure was and how should my package be used, but I shared few about \LaTeX. My package and workflow simplified some concepts so that the team didn't need to care about some parts of \LaTeX{} such as the documentclass and preamble, though \LaTeX{} itself is still a lot different from Word. I underestimated the difficulty for my colleagues to learn basic \LaTeX{} typing skills so it was a torment for some of them. I some kind of realized this problem in the later stages, but I was occupied by my own businesses, and didn't pay enough attention to it.

The final results were generally successful and mostly good enough. Yet I feel sorry for my colleagues because this thing was in fact not that easy for them. In addition, I recently took part in the 15\textsuperscript{th} anniversary of XJTLU math club and this somehow strengthened my sense of belonging to the math club. For these reasons, I decided to leave something for XJTLU's students in the future. I rewrote the whole package and added a series of tutorials that covers the basics of \LaTeX{} as well as making materials, so that people in the material dept.~will do much less to make a material in \LaTeX. The tutorials are in the form of a stories, since I think this method adds more fun to the studying.

Apart from that, the general students of XJTLU or even other \LaTeX{} users may find this package and documentation useful. So its final form was decided to be a template with documentation that is published as a GitHub repository under my personal account. This is because the math club it self doesn't have a GitHub account currently. I will transfer this repository to its official account once it has established one.